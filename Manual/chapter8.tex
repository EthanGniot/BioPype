\chapter{Software and Set-up}
\label{chap:software}
\section{Software}
(Table of software name, name in PATH, version number, function of the software for each one we're gonna use)
\todo{Add further software that you end up using (e.g., USEARCH). ALSO, make sure to update the section that mentions installing packages if you do so.}
\todo[inline]{write about installing qiime2 \url{https://docs.qiime2.org/2018.4/install/native/}}
\todo[inline]{write about installing parallel-fastq-dump}
\todo[inline]{write about installing multifastqc}
\todo[inline]{add 3rd column to table with link to documentation/download source}
%
\begin{table}[htp]
    \begin{center}
    \begin{tabular}{ l | l }
        \textit{Software name} & \textit{Version number} \\ 
        \hline
        Anaconda & 5.1 \\  
        Biopython & 1.70 \\
        BLAT & 35 \\
        matplotlib & 2.2.2 \\
        pandas & 0.22.0 \\
        sra-tools & 2.8.2 \\
        trim-galore & 0.4.5 \\
    \end{tabular}
    \caption{\textbf{Software used to create the tutorial pipeline.}}
    \label{tab:software}
    \end{center}
    \label{software}
\end{table}
%
\section{Set-up and Install Dependencies}
Before we write any code, there are several steps that must be completed to prep your machine for the tasks we will be performing in this tutorial. Without these prerequisites, the code you write during this tutorial will not work correctly:
\begin{enumerate}
\item Install and open Anaconda
\item Create a new virtual environment
\item Install packages
\end{enumerate}

\subsection{Install Anaconda}
The Anaconda program will play a key role in this tutorial. Anaconda is essentially Python and a lot of scientific computing tools bundled together, along with many popular add-ons to Python called packages. \seealso{RESOURCE FOR LEARNING ABOUT PYTHON PACKAGES} Downloading all of these tools individually can be difficult, as the quirks of one package may conflict with another when they're installed manually; using Anaconda to install packages greatly simplifies the process because Anaconda can smoothly handle all of the minute details that cause manual installations to fail.

\textbf{\underline{Install and Open:}}

\begin{enumerate}
    \item Go to the download page for the Anaconda distribution at \\ \url{https://www.anaconda.com/download}. 
    \item Select your preferred operating system from the Windows, macOS, or Linux tabs, then select the Download option for the \textbf{Python 3.6 version} (Figure \ref{anaconda_download}) and follow the installation instructions.
    %
    \begin{figure}[h]
        \begin{center}
        \includegraphics[width=9.5cm]{anaconda_download}
        \caption{The Anaconda download options provided on the Anaconda distribution website at \protect \url{https://www.anaconda.com/download}}.
        \label{anaconda_download}
        \end{center}
    \end{figure}
    %


    \item After installation is complete, open the application named "Anaconda-Navigator" (the icon looks like \includegraphics[width=0.5cm]{anaconda-navigator-thumbnail}). After a brief start-up period, you should see the following window (Figure \ref{anaconda-nav-win}):
    %
    \begin{figure}[htbp]
        \begin{center}
        \includegraphics[width=12cm]{anaconda-nav-win}
        \caption{The window displayed to the user upon opening Anaconda-Navigator.}
        \label{anaconda-nav-win}
        \end{center}
    \end{figure}
    %
\end{enumerate}

\subsection{Create a New Virtual Environment}
    \todo[inline]{Write a blurb explaining the benefits of using a virtual environment.}
    \todo[inline]{Link to resource for further reading on virtual environments}
    \marginlabel{Make sure the computer has an internet connection while completing this section, otherwise Anaconda will not let you create a virtual environment.}
    \begin{enumerate}
        \item On the left side of the Anaconda-Navigator window, click on the tab labeled \textbf{Environments}. (Figure \ref{anaconda-env-win}) 
        %
        \begin{figure}
            \includegraphics[width=12cm]{anaconda-env-win}
            \caption{The Environments window of the Anaconda-Navigator.}
            \label{anaconda-env-win}
        \end{figure}
        %
        \item Click the \textbf{Create} button on the bottom of the center panel. A new window titled "Create new environment" will appear. (Figure \ref{anaconda-create-new-env-win})
        %
        \begin{figure}[h]
            \begin{center}
            \includegraphics[width=12cm]{anaconda-create_new_env_win}
            \caption{The "Create new environment" window.}
            \label{anaconda-create-new-env-win}
            \end{center}
        \end{figure}
        %
        \item Enter a \textbf{Name} for the environment. You may choose any name you want, but for the sake of this tutorial we will name the new environment "BioPype".
        \item Select the box labeled \textbf{Python} next to the \textbf{Packages} heading.
        \item Choose a version of Python from the adjacent drop-down menu (Python 3.6 is the most current version at the time of this writing, but the packages we use require Python 3.5 so we chose \textbf{3.5}. If you are following the tutorial analysis in this manual, choose version 3.5).
        \item Click the \textbf{Create} button within the "Create new environment window".
    \end{enumerate}

\subsection{Install packages}
    \todo[inline]{Write blurb about what packages are.}
    \todo[inline]{Link to resource for further reading about packages.}
    \begin{enumerate}
        \item Change Anaconda's current environment from the \textbf{root} environment by selecting the \textbf{BioPype} tab in the middle panel of the Environments window.
        \item Click on the drop-down menu in the right-hand panel that says "Installed" and change it to "All".
        \item In the "Search Packages" box, enter "biopython". The search should return a package named "biopython". Select the checkbox to the left of the name. (Figure \ref{anaconda-search-pack})
        \begin{itemize}
            \item A pair of green and red boxes (reading "Apply" and "Clear", respectively) will appear in the bottom-right of the window once the package is selected. Do not click these just yet. 
    %
    \begin{figure}[hbtp]
        \begin{center}
        \includegraphics[width=12cm]{anaconda-search-pack}
        \caption{Searching for a package. When a package is selected, the checkbox next to the package's name will be green.}
        \label{anaconda-search-pack}
        \end{center}
    \end{figure}
    %
    \todo{Solve the issue with failing to center figures when they're on their own page}
        \end{itemize}
        \item Use the search bar to find and select the other packages listed in \autoref{tab:software}. Once all packages have been selected, click the green "Apply" button in the bottom right corner of the window, then select "Apply" again within the "Install Packages" window that appears. (Figure \ref{anaconda-install-pack}) Anaconda will now install the selected packages.
    %
    \begin{figure}[hbtp]
        \begin{center}
        \includegraphics[width=12cm]{anaconda-install-pack}
        \caption{The window displaying the packages and dependencies that will be installed.}
        \label{anaconda-install-pack}
        \end{center}
    \end{figure}
    %
        \item \todo[inline]{Talk about setting up the sra-tools workspace (\url{https://trace.ncbi.nlm.nih.gov/Traces/sra/sra.cgi?view=toolkit_doc&f=std})}
        \item \todo[inline]{PYPIPER IS ONLY COMPATIBLE WITH MAC AND LINUX. Start by coding with just subprocess module commands. If the scripts work on Windows computers, then forget about using pypiper. But if the subprocess scripts don't work on windows, then we'll be developing exclusively for Mac anyways, so you could use pypiper without any worries. In that case, talk about installing pypiper here. OTHERWISE, delete this section.}
        \begin{enumerate}
            \item Open a terminal window in the BioPype environment by clicking the "play" button on the BioPype environment tab and then selecting "Open Terminal".
            \item Wait for the terminal window to finish opening. You'll know it's finished when you see \todo[inline]{show image of what the prompt looks like when it's done initializing.}
            \item Install \textbf{pypiper} by typing the following at the command prompt, followed by pressing return/enter:
            %
            \begin{lstlisting} [language=Python]
                pip install --user pypiper
            \end{lstlisting}
            %
            \item Check that the package was installed correctly by executing the following in the command prompt:
            %
            \begin{lstlisting} [language=Python]
                conda list
            \end{lstlisting}
            %
            This will generate a list of all the packages installed in the current environment. If you see the \textbf{pypiper} package listed, the installation was successful and you may skip the rest of this section. If not, proceed with the following steps.
            \item Execute the install command from step (c) again. This time, the Terminal should return a message similar to the one displayed in \todo[inline]{Reference the figure pypiper-already-installed}. The line that reads "Requirement already satisfied: pypiper in ...." tells us the location where the package was (incorrectly) installed. 
            \missingfigure{pypiper-already-installed}
            \item Open a Terminal window, and navigate to the location indicated by the message from the previous step. For my example, I need to start at my home directory and walk through the following folders: .local | lib | python3.6 | site-packages. 
            \begin{itemize}
                \item The folders along the path to the pypiper installation may be hidden. On a Mac, these hidden folders are preceded by a "." If the path to the pypiper installation includes hidden locations, reveal them by pressing "Cmd + Shift + ." in the Finder window.
            \item Once you find the site-packages folder containing two pypiper folders \todo[inline]{reference figure pypiper-wrong-location}, copy those folders and their contents and paste them into the ~/anaconda/envs/BioPype/lib/python3.6/site-packages directory. The package should now be installed correctly. \missingfigure{pypiper-wrong-location} 
            \marginlabel{}
            \end{itemize}
        \end{enumerate}    
    \end{enumerate}

\subsection{Integrated Development Environment (IDE)}
\todo[inline]{Talk about choosing between PyCharm and other options}
%
\subsection{PATH}
\todo[inline]{(Talk about putting all the tools in the same path/directory)}
\section{Analysis Pipeline}
(Use figures to illustrate the stages of the pipeline)
